% در این فایل، عنوان پایان‌نامه، مشخصات خود، متن تقدیمی‌، ستایش، سپاس‌گزاری و چکیده پایان‌نامه را به فارسی، وارد کنید.
% توجه داشته باشید که جدول حاوی مشخصات پروژه/پایان‌نامه/رساله و همچنین، مشخصات داخل آن، به طور خودکار، درج می‌شود.
%%%%%%%%%%%%%%%%%%%%%%%%%%%%%%%%%%%%
% دانشگاه خود را وارد کنید
\university {الزهراء(س)}
% دانشکده، آموزشکده و یا پژوهشکده  خود را وارد کنید (به فارسی و انگلیسی).
\faculty{ دانشکده علوم ریاضی}
\enfaculty{Mathematical Sciences}

% گروه آموزشی خود را وارد کنید (به فارسی و انگلیسی).
\department{ریاضی  }
\endepartment{Mathematics}


% گروه آموزشی خود را وارد کنید (به فارسی و انگلیسی).
\subject{ریاضی کاربردی}
\ensubject{Applied Mathematics}

% گرایش خود را وارد کنید (به فارسی و انگلیسی). در صورتی که فقط رشته تحصیلی دارید مانند رشته بیوانفورماتیک باید این دستور را با قرار دادن علامت % قبل از آن غیر فعال کنید.
\field{بهینه‌سازی} 
\enfield{Optimization}

% عنوان پایان‌نامه را وارد کنید (به فارسی و انگلیسی).
\title{
نوشتن پروژه، پایان‌نامه و رساله با استفاده از کلاس 
\lr{AlzahraThesis} 
}
\entitle{Writing projects, theses and dissertations using Alzahra\_thesis Class}

% نام استاد(ان) راهنما را وارد کنید (به فارسی و انگلیسی).
\firstsupervisor{فرید بهروزی}
\enfirstsupervisor{Farid Behrouzi}

% وابستگی سازمانی استاد راهنمای اول را وارد کنید. توجه کنید که وابستگی سازمان در قالب زیر بیان شود:
% گروه، دانشکده، دانشگاه، شهر، کشور
\firstsupervisoraffiliation{%
گروه ریاضی، دانشکده علوم ریاضی، دانشگاه الزهرا (س)، تهران، ایران
}
\enfirstsupervisoraffiliation{%
Department of Mathematics, Faculty of Mathematical Sciences, Alzahra Univesrity, Tehran, Iran
}

\secondsupervisor{ابولقاسم لاله}
\ensecondsupervisor{Abolghasem Laleh}

\secondsupervisoraffiliation{%
گروه ریاضی، دانشکده علوم ریاضی، دانشگاه الزهرا (س)، تهران، ایران
}
\ensecondsupervisoraffiliation{Department of Mathematics, Faculty of Mathematical Sciences, Alzahra Univesrity, Tehran, Iran
}



% نام استاد(دان) مشاور را وارد کنید. چنانچه استاد مشاور ندارید، دستور پایین را غیرفعال کنید.
\firstadvisor{علیمردان شاهرضایی}
\enfirstadvisor{Alimardan Shahrezaee}

\firstadvisoraffiliation{%
گروه ریاضی، دانشکده علوم ریاضی، دانشگاه الزهرا (س)، تهران، ایران
}
\enfirstadvisoraffiliation{Department of Mathematics, Faculty of Mathematical Sciences, Alzahra Univesrity, Tehran, Iran
}

\secondadvisor{کامران دیوانی آذر}
\ensecondadvisor{Kamran Divaniazar}

\secondadvisoraffiliation{%
گروه ریاضی، دانشکده علوم ریاضی، دانشگاه الزهرا (س)، تهران، ایران
}
\ensecondadvisoraffiliation{Department of Mathematics, Faculty of Mathematical Sciences, Alzahra Univesrity, Tehran, Iran
}



% نام پژوهشگر را وارد کنید
\name{احسان}
\enname{Sayyed Ehsan}
% نام خانوادگی پژوهشگر را وارد کنید
\surname{منبتی}
\ensurname{Monabbati}

% تاریخ پایان‌نامه را وارد کنید
\thesisdate{ فروردین‌ماه ۱۳۹۵}
\enthesisdate{March, 2011 }

% کلمات کلیدی پایان‌نامه را وارد کنید
\keywords{بهینه‌سازی نامقید، روش گرادیان مزدوج}
\enkeywords{Probabilistic powerdomain; Stably compact space; Valuation}


% چکیده پایان‌نامه را وارد کنید
\faAbstract{
 چکیده می­تواند 250 تا 300 واژه داشته باشد و باید در یک صفحه گنجانده شود. در نگارش چکیده جمله‌های کامل (فعل‌دار) به‌کار می‌روند. چکیده شامل هدف، روش­شناسی پژوهش، یافته­ها، نتیجه‌گیری وکلیدواژه­ها است. در چکیده، بیشتر جمله‌های معلوم به‌جای جمله‌های مجهول می‌آیند. در چکیده فرمولی نوشته نمی‌شود ولی اگر نیاز باشد باید واژه‌های فارسی برای نوشتن آن به‌کار روند. در متن چکيده، از ارجاع به منابع و اشاره به جداول و نمودارها اجتناب شود. در پایان‌نامه/ رساله­هایی که متن اصلی آن‌ها به زبان فارسی است، نخست چکیده فارسی و در پایان نیز چکیده به زبان انگلیسی می‌آید، پایان­نامه/ رساله­هایی که به سایر زبان­ها (عربی، فرانسه و...) نوشته می‌شوند، چکیده باید به سه زبان، 1- زبان فارسی (برای خواننده‌های فارسی‌زبان)، 2- سایر زبان‌های عربی و فرانسه و ... (برای خواننده‌های عربی زبان، فرانسه زبان و ...)، 3- زبان انگلیسی نوشته شود. همچنین ترتیب قرارگیری چکیده‌ها در پایان‌نامه/ رساله‌هایی که متن اصلی آن‌ها به سایر زبان‌ها (عربی و فرانسه و ..) است به‌صورت زیر می‌باشد: 1- چکیده به سایر زبان‌ها (عربی و فرانسه و ...)، 2- چکیده به زبان فارسی در ابتدای پایان­نامه، 3- چکیده به زبان انگلیسی در انتهای پایان‌نامه/ رساله. 
}

\enAbstract{
Abstracts can be 250 to 300 words long and must fit on a single page. Complete sentences (with verbs) should be used when writing the abstract. The abstract should include the research aim, methodology, findings, conclusion, and keywords. In the abstract, active sentences are preferred over passive sentences. No formulas should be included in the abstract, but if necessary, Persian words should be used to write them. References to sources, tables, and figures should be avoided in the abstract. In theses/dissertations written in Persian, the Persian abstract comes first, followed by an English abstract at the end. For theses/dissertations written in other languages (Arabic, French, etc.), the abstract must be provided in three languages: 1- Persian (for Persian-speaking readers), 2- Other languages, such as Arabic or French (for Arabic-speaking, French-speaking readers, etc.), and 3- English. The order of abstracts in theses/dissertations written in other languages (Arabic, French, etc.) should be as follows: 1- Abstract in other languages (Arabic, French, etc.), 2- Abstract in Persian at the beginning of the thesis, and 3- Abstract in English at the end of the thesis/dissertation.
}

% توجه کنید که نمره پایان‌نامه را فقط بعد از دفاعیه در این بخش وارد کنید. 
\defencemark{ }

% اگر پایان‌نامه / رساله شما با حمایت مالی سازمانی انجام شده است که موافقت معاونت پژوهشی دانشگاه الزهرا و سازمان حمایت‌کننده را دارند در دستور زیر نام سازمان حمایت‌کننده را ذکر کنید. در غیر این صورت دستور زیر را غیرفعال کنید.
\supportedby{ }

%توضیح اضافه شود
%\shortbio{
%........ دانش‌آموخته مقطع .......................رشته ........................... از دانشگاه .................... درگرایش ........... در سال ................. است. او در سال ....... مدرک مقطع ............. خود را از دانشگاه ................... در رشته ................. گرایش .............. و مدرک مقطع ............. خود را در سال ......... از دانشگاه .................. در رشته ................................ دریافت کرد. زمینه‌های پژوهشی وی عبارتند از 
%}
%
%\enshortbio{
%……………[NAME]…………………. has obtained her …[DEGREE]….. degree in the field of …………[DISCIPLINE]…………..… and the sub-discipline of …….…[SUB-DISCIPLINE, IF APPLICABLE]…….…….. from Alzahra University in the year ..[YEAR]..  Former to that, she obtained her ……[DEGREE]…… degree in the field of …………[DISCIPLINE]………….... and the sub-discipline of …………[SUB-DISCIPLINE, IF APPLICABLE]……..… from ………[UNIVERSITY]……….. in the year …[YEAR]….,  and her …[DEGREE]…. degree in the field of …………[DISCIPLINE]……….….. and the sub-discipline of ………[SUB-DISCIPLINE, IF APPLICABLE].…… from ………[UNIVERSITY]………… in the year .…[YEAR]……  Her research interests include: 
%}




