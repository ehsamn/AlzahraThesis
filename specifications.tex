% در این فایل، عنوان پایان‌نامه، مشخصات خود، متن تقدیمی‌، ستایش، سپاس‌گزاری و چکیده پایان‌نامه را به فارسی، وارد کنید.
% توجه داشته باشید که جدول حاوی مشخصات پروژه/پایان‌نامه/رساله و همچنین، مشخصات داخل آن، به طور خودکار، درج می‌شود.
%%%%%%%%%%%%%%%%%%%%%%%%%%%%%%%%%%%%
% دانشگاه خود را وارد کنید
\university {الزهراء(س)}
% دانشکده، آموزشکده و یا پژوهشکده  خود را وارد کنید
\faculty{ دانشکده علوم ریاضی}
% گروه آموزشی خود را وارد کنید
\department{ریاضی  }
% گروه آموزشی خود را وارد کنید
\subject{ریاضی محض}
% گرایش خود را وارد کنید
 \field{ آنالیز ریاضی} 
% عنوان پایان‌نامه را وارد کنید
\title{
نوشتن پروژه، پایان‌نامه و رساله با استفاده از کلاس 
\lr{AlzahraThesis} 
}
% نام استاد(ان) راهنما را وارد کنید
\firstsupervisor{فرید بهروزی}
%\secondsupervisor{ابولقاسم لاله}

% وابستگی سازمانی استاد راهنمای اول را وارد کنید. توجه کنید که وابستگی سازمان در قالب زیر بیان شود:
% گروه، دانشکده، دانشگاه، شهر، کشور
\firstsupervisoraffiliation{%
گروه ریاضی، دانشکده علوم ریاضی، دانشگاه الزهرا (س)، تهران، ایران
}
\secondsupervisoraffiliation{%
گروه ریاضی، دانشکده علوم ریاضی، دانشگاه الزهرا (س)، تهران، ایران
}
% نام استاد(دان) مشاور را وارد کنید. چنانچه استاد مشاور ندارید، دستور پایین را غیرفعال کنید.
\firstadvisor{علیمردان شاهرضایی}
\firstadvisoraffiliation{%
گروه ریاضی، دانشکده علوم ریاضی، دانشگاه الزهرا (س)، تهران، ایران
}
\secondadvisor{کامران دیوانی آذر}
\secondadvisoraffiliation{%
گروه ریاضی، دانشکده علوم ریاضی، دانشگاه الزهرا (س)، تهران، ایران
}
% نام پژوهشگر را وارد کنید
\name{احسان}
% نام خانوادگی پژوهشگر را وارد کنید
\surname{منبتی}
% تاریخ پایان‌نامه را وارد کنید
\thesisdate{ فروردین‌ماه ۱۳۹۵}
% کلمات کلیدی پایان‌نامه را وارد کنید
\keywords{ارزیابی، دامنه‌توانی احتمالی، فضای فشرده پایدار}
% چکیده پایان‌نامه را وارد کنید
\faAbstract{
توضیح: چکیده می­تواند 250 تا 300 واژه داشته باشد و باید در یک صفحه گنجانده شود. در نگارش چکیده جمله‌های کامل (فعل‌دار) به‌کار می‌روند. چکیده شامل هدف، روش­شناسی پژوهش، یافته­ها، نتیجه‌گیری وکلیدواژه­ها است. در چکیده، بیشتر جمله‌های معلوم به‌جای جمله‌های مجهول می‌آیند. در چکیده فرمولی نوشته نمی‌شود ولی اگر نیاز باشد باید واژه‌های فارسی برای نوشتن آن به‌کار روند. در متن چکيده، از ارجاع به منابع و اشاره به جداول و نمودارها اجتناب شود. در پایان‌نامه/ رساله­هایی که متن اصلی آن‌ها به زبان فارسی است، نخست چکیده فارسی و در پایان نیز چکیده به زبان انگلیسی می‌آید، پایان­نامه/ رساله­هایی که به سایر زبان­ها (عربی، فرانسه و...) نوشته می‌شوند، چکیده باید به سه زبان، 1- زبان فارسی (برای خواننده‌های فارسی‌زبان)، 2- سایر زبان‌های عربی و فرانسه و ... (برای خواننده‌های عربی زبان، فرانسه زبان و ...)، 3- زبان انگلیسی نوشته شود. همچنین ترتیب قرارگیری چکیده‌ها در پایان‌نامه/ رساله‌هایی که متن اصلی آن‌ها به سایر زبان‌ها (عربی و فرانسه و ..) است به‌صورت زیر می‌باشد: 1- چکیده به سایر زبان‌ها (عربی و فرانسه و ...)، 2- چکیده به زبان فارسی در ابتدای پایان­نامه، 3- چکیده به زبان انگلیسی در انتهای پایان‌نامه/ رساله. 
متنی برای امتحان (\lr{Test})
}
% توضیح اضافه شود
\defencemark{خیلی خوب}

% اگر پایان‌نامه / رساله شما با حمایت مالی سازمانی انجام شده است که موافقت معاونت پژوهشی دانشگاه الزهرا و سازمان حمایت‌کننده را دارند در دستور زیر نام سازمان حمایت‌کننده را ذکر کنید. در غیر این صورت دستور زیر را غیرفعال کنید.
\supportedby{شهرداری تهران}

%توضیح اضافه شود
\shortbio{
........ دانش‌آموخته مقطع .......................رشته ........................... از دانشگاه .................... درگرایش ........... در سال ................. است. او در سال ....... مدرک مقطع ............. خود را از دانشگاه ................... در رشته ................. گرایش .............. و مدرک مقطع ............. خود را در سال ......... از دانشگاه .................. در رشته ................................ دریافت کرد. زمینه‌های پژوهشی وی عبارتند از 
}

\enshortbio{
……………[NAME]…………………. has obtained her …[DEGREE]….. degree in the field of …………[DISCIPLINE]…………..… and the sub-discipline of …….…[SUB-DISCIPLINE, IF APPLICABLE]…….…….. from Alzahra University in the year ..[YEAR]..  Former to that, she obtained her ……[DEGREE]…… degree in the field of …………[DISCIPLINE]………….... and the sub-discipline of …………[SUB-DISCIPLINE, IF APPLICABLE]……..… from ………[UNIVERSITY]……….. in the year …[YEAR]….,  and her …[DEGREE]…. degree in the field of …………[DISCIPLINE]……….….. and the sub-discipline of ………[SUB-DISCIPLINE, IF APPLICABLE].…… from ………[UNIVERSITY]………… in the year .…[YEAR]……  Her research interests include: 
}

\enfaculty{Mathematical Sciences}
\endepartment{Mathematics}
\ensubject{Applied Mathematics}
\enfield{Operations Research}
\endepartment{Applied Mathematics}
\entitle{Writing projects, theses and dissertations using Alzahra\_thesis Class}
\enfirstsupervisor{Farid Behrouzi}
\enfirstsupervisoraffiliation{%
Department of Mathematics, Faculty of Mathematical Sciences, Alzahra Univesrity, Tehran, Iran
}

\ensecondsupervisor{Abolghasem Laleh}
\ensecondsupervisoraffiliation{Department of Mathematics, Faculty of Mathematical Sciences, Alzahra Univesrity, Tehran, Iran
}
\enfirstadvisor{Alimardan Shahrezaee}
\enfirstadvisoraffiliation{Department of Mathematics, Faculty of Mathematical Sciences, Alzahra Univesrity, Tehran, Iran
}
\ensecondadvisor{Kamran Divaniazar}
\ensecondadvisoraffiliation{Department of Mathematics, Faculty of Mathematical Sciences, Alzahra Univesrity, Tehran, Iran
}

\enname{Ehsan}
\ensurname{Monabbati}
\enthesisdate{March, 2011 }
\enkeywords{Probabilistic powerdomain; Stably compact space; Valuation}

\enAbstract{
This thesis is about typsetting using LaTex
}



