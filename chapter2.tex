\chapter{فضاهای متریک }
\thispagestyle{empty}

\section{مقدمه }
با توجه به تعریف ارائه شده از یک  فضای متری در  
\cite{rud}


\begin{equation}
a = b
\end{equation}
\begin{equation}
a = b
\end{equation}


\section{فضاهای متریک فشرده }
در این  بخش به بیان ...
ز جملهٔ کارآترین ابزار و شیوه‌های گسترش و پیشرفت در ریاضیات (و در بسیاری از میدان‌ها و زمینه‌های دیگر حیات انسانی) تجرید، و از آن هم مهم‌تر، تعمیم است.
اکبر هاشمی بهرمانی (زادهٔ ۳ شهریور ۱۳۱۳ در رفسنجان – درگذشتهٔ ۱۹ دی ۱۳۹۵ در تهران) مشهور بهعلی اکبر هاشمی رفسنجانی روحانی، مفسر قرآن،[۱۱][۱۲] و سیاستمدار ایرانی بود که به عنوان یکی از پرنفوذترین شخصیت‌های سیاسی جمهوری اسلامی شناخته می‌شد. او از ۲۵ مرداد ۱۳۶۸ تا ۱۲ مرداد ۱۳۷۶ چهارمین رئیس‌جمهور ایران بود. پیش از آن وی اولین رئیس مجلس شورای اسلامی از سال ۱۳۵۹ تا ۱۳۶۸ بود. در طول جنگ ایران و عراق رفسنجانی به عنوان نماینده رهبر در شورای عالی دفاع عملاً فرماندهی نیروهای مسلح را برعهده داشت. رفسنجانی از سال ۱۳۶۸ تا زمان مرگ، ریاست مجمع تشخیص مصلحت نظام را برعهده داشت. او در همه ادوار مجلس خبرگان رهبری نمایندگی استان تهران در آن را بر عهده داشت و از ۱۳۸۶ تا ۱۳۹۰، دومین رئیس این مجلس بود. رفسنجانی رئیس هیئت مؤسس و هیئت امنای دانشگاه آزاد اسلامی بود.
او یکی از تعیین‌کننده‌ترین سیاستمداران ایران، دوست نزدیک سید علی خامنه‌ای و شاگرد روح‌الله خمینی و رازدار اسرار جمهوری اسلامی ایران بود.[۱۳] در دوران ریاست جمهوری، با توجه به شرایط جدید کشور و پایان جنگ، وی اقدامات گسترده‌ای را در طی ۸ سال در ابعاد گوناگون سیاست داخلی و خارجی، اقتصادی، نظامی و امنیتی انجام داد که به دوران سازندگی شهرت یافته‌است. مهم‌ترین اولویت دولت در این دوران، بازسازی زیربناهای اقتصادی و صنعتی کشور، توسعه روابط سیاست خارجی و اصلاح ساختار مدیریتی کشور بود که انتقاداتی را نیز در پی داشته‌است.[۱۴] در این دوران، تعداد بسیار زیادی سد، کارخانه و کارگاه تولید صنعتی راه‌اندازی، و در بسیاری از شهرهای کوچک و بزرگ دانشگاه ساخته گردید.[۱۵] این اصلاحات، تبعاتی منفی نظیر افزایش تورم و قیمت مسکن در دورهٔ اول ریاست جمهوری وی را نیز در پی داشت. ثبت رکورد تورم ۴۹٫۴ درصدی بالاترین رقم تورم ثبت شده در تاریخ ایران، در دوران ریاست جمهوری وی اتفاق افتاد.[۱۶]
هاشمی رفسنجانی در سال ۱۳۸۴ در انتخابات ریاست جمهوری شرکت کرد که در دور دوم از محمود احمدی‌نژاد شکست خورد. در جریان انتخابات ریاست جمهوری ۱۳۸۸ هاشمی و خانواده او به وضوح ازمیرحسین موسوی، کاندیدای اصلاح‌طلبان حمایت کردند و خود او نیز در آخرین نمازجمعه‌ای که در تهران اقامه کرد و پس از آن در محافل خصوصی از سرکوب معترضان به نتایج این انتخابات انتقاد نمود. این انتقادات باعث شد که از سوی مراجع حکومتی به همسویی با معترضان متهم شده و به تدریج جایگاه امامت نمازجمعه، ریاست مجلس خبرگان و اثرگذاری بر دانشگاه آزاد اسلامی را از دست داده و فرزندانش نیز پست‌های خود را از دست بدهند.[۱۷]
رفسنجانی در سال ۱۳۹۲ چهارمین بار در انتخابات ریاست جمهوری نامزد شد که صلاحیت وی از سویشورای نگهبان احراز نشد. با پیروزی حسن روحانی در انتخابات، با توجه به رابطه نزدیک او با هاشمی رفسنجانی و حمایت هاشمی از وی در جریان انتخابات، خانواده هاشمی رفسنجانی به تدریج وجهه سیاسی خود را بازیافتند و به سرعت رئیس دانشگاه آزاد اسلامی تغییر کرد. او در سال‌های پایانی عمر خود توانست محبوبیتی مکدر در میان ایرانیان پیدا کند.[۱۸
اکبر هاشمی بهرمانی (زادهٔ ۳ شهریور ۱۳۱۳ در رفسنجان – درگذشتهٔ ۱۹ دی ۱۳۹۵ در تهران) مشهور بهعلی اکبر هاشمی رفسنجانی روحانی، مفسر قرآن،[۱۱][۱۲] و سیاستمدار ایرانی بود که به عنوان یکی از پرنفوذترین شخصیت‌های سیاسی جمهوری اسلامی شناخته می‌شد. او از ۲۵ مرداد ۱۳۶۸ تا ۱۲ مرداد ۱۳۷۶ چهارمین رئیس‌جمهور ایران بود. پیش از آن وی اولین رئیس مجلس شورای اسلامی از سال ۱۳۵۹ تا ۱۳۶۸ بود. در طول جنگ ایران و عراق رفسنجانی به عنوان نماینده رهبر در شورای عالی دفاع عملاً فرماندهی نیروهای مسلح را برعهده داشت. رفسنجانی از سال ۱۳۶۸ تا زمان مرگ، ریاست مجمع تشخیص مصلحت نظام را برعهده داشت. او در همه ادوار مجلس خبرگان رهبری نمایندگی استان تهران در آن را بر عهده داشت و از ۱۳۸۶ تا ۱۳۹۰، دومین رئیس این مجلس بود. رفسنجانی رئیس هیئت مؤسس و هیئت امنای دانشگاه آزاد اسلامی بود.
او یکی از تعیین‌کننده‌ترین سیاستمداران ایران، دوست نزدیک سید علی خامنه‌ای و شاگرد روح‌الله خمینی و رازدار اسرار جمهوری اسلامی ایران بود.[۱۳] در دوران ریاست جمهوری، با توجه به شرایط جدید کشور و پایان جنگ، وی اقدامات گسترده‌ای را در طی ۸ سال در ابعاد گوناگون سیاست داخلی و خارجی، اقتصادی، نظامی و امنیتی انجام داد که به دوران سازندگی شهرت یافته‌است. مهم‌ترین اولویت دولت در این دوران، بازسازی زیربناهای اقتصادی و صنعتی کشور، توسعه روابط سیاست خارجی و اصلاح ساختار مدیریتی کشور بود که انتقاداتی را نیز در پی داشته‌است.[۱۴] در این دوران، تعداد بسیار زیادی سد، کارخانه و کارگاه تولید صنعتی راه‌اندازی، و در بسیاری از شهرهای کوچک و بزرگ دانشگاه ساخته گردید.[۱۵] این اصلاحات، تبعاتی منفی نظیر افزایش تورم و قیمت مسکن در دورهٔ اول ریاست جمهوری وی را نیز در پی داشت. ثبت رکورد تورم ۴۹٫۴ درصدی بالاترین رقم تورم ثبت شده در تاریخ ایران، در دوران ریاست جمهوری وی اتفاق افتاد.[۱۶]
هاشمی رفسنجانی در سال ۱۳۸۴ در انتخابات ریاست جمهوری شرکت کرد که در دور دوم از محمود احمدی‌نژاد شکست خورد. در جریان انتخابات ریاست جمهوری ۱۳۸۸ هاشمی و خانواده او به وضوح ازمیرحسین موسوی، کاندیدای اصلاح‌طلبان حمایت کردند و خود او نیز در آخرین نمازجمعه‌ای که در تهران اقامه کرد و پس از آن در محافل خصوصی از سرکوب معترضان به نتایج این انتخابات انتقاد نمود. این انتقادات باعث شد که از سوی مراجع حکومتی به همسویی با معترضان متهم شده و به تدریج جایگاه امامت نمازجمعه، ریاست مجلس خبرگان و اثرگذاری بر دانشگاه آزاد اسلامی را از دست داده و فرزندانش نیز پست‌های خود را از دست بدهند.[۱۷]
رفسنجانی در سال ۱۳۹۲ چهارمین بار در انتخابات ریاست جمهوری نامزد شد که صلاحیت وی از سویشورای نگهبان احراز نشد. با پیروزی حسن روحانی در انتخابات، با توجه به رابطه نزدیک او با هاشمی رفسنجانی و حمایت هاشمی از وی در جریان انتخابات، خانواده هاشمی رفسنجانی به تدریج وجهه سیاسی خود را بازیافتند و به سرعت رئیس دانشگاه آزاد اسلامی تغییر کرد. او در سال‌های پایانی عمر خود توانست محبوبیتی مکدر در میان ایرانیان پیدا کند.[۱۸
اکبر هاشمی بهرمانی (زادهٔ ۳ شهریور ۱۳۱۳ در رفسنجان – درگذشتهٔ ۱۹ دی ۱۳۹۵ در تهران) مشهور بهعلی اکبر هاشمی رفسنجانی روحانی، مفسر قرآن،[۱۱][۱۲] و سیاستمدار ایرانی بود که به عنوان یکی از پرنفوذترین شخصیت‌های سیاسی جمهوری اسلامی شناخته می‌شد. او از ۲۵ مرداد ۱۳۶۸ تا ۱۲ مرداد ۱۳۷۶ چهارمین رئیس‌جمهور ایران بود. پیش از آن وی اولین رئیس مجلس شورای اسلامی از سال ۱۳۵۹ تا ۱۳۶۸ بود. در طول جنگ ایران و عراق رفسنجانی به عنوان نماینده رهبر در شورای عالی دفاع عملاً فرماندهی نیروهای مسلح را برعهده داشت. رفسنجانی از سال ۱۳۶۸ تا زمان مرگ، ریاست مجمع تشخیص مصلحت نظام را برعهده داشت. او در همه ادوار مجلس خبرگان رهبری نمایندگی استان تهران در آن را بر عهده داشت و از ۱۳۸۶ تا ۱۳۹۰، دومین رئیس این مجلس بود. رفسنجانی رئیس هیئت مؤسس و هیئت امنای دانشگاه آزاد اسلامی بود.
او یکی از تعیین‌کننده‌ترین سیاستمداران ایران، دوست نزدیک سید علی خامنه‌ای و شاگرد روح‌الله خمینی و رازدار اسرار جمهوری اسلامی ایران بود.[۱۳] در دوران ریاست جمهوری، با توجه به شرایط جدید کشور و پایان جنگ، وی اقدامات گسترده‌ای را در طی ۸ سال در ابعاد گوناگون سیاست داخلی و خارجی، اقتصادی، نظامی و امنیتی انجام داد که به دوران سازندگی شهرت یافته‌است. مهم‌ترین اولویت دولت در این دوران، بازسازی زیربناهای اقتصادی و صنعتی کشور، توسعه روابط سیاست خارجی و اصلاح ساختار مدیریتی کشور بود که انتقاداتی را نیز در پی داشته‌است.[۱۴] در این دوران، تعداد بسیار زیادی سد، کارخانه و کارگاه تولید صنعتی راه‌اندازی، و در بسیاری از شهرهای کوچک و بزرگ دانشگاه ساخته گردید.[۱۵] این اصلاحات، تبعاتی منفی نظیر افزایش تورم و قیمت مسکن در دورهٔ اول ریاست جمهوری وی را نیز در پی داشت. ثبت رکورد تورم ۴۹٫۴ درصدی بالاترین رقم تورم ثبت شده در تاریخ ایران، در دوران ریاست جمهوری وی اتفاق افتاد.[۱۶]
هاشمی رفسنجانی در سال ۱۳۸۴ در انتخابات ریاست جمهوری شرکت کرد که در دور دوم از محمود احمدی‌نژاد شکست خورد. در جریان انتخابات ریاست جمهوری ۱۳۸۸ هاشمی و خانواده او به وضوح ازمیرحسین موسوی، کاندیدای اصلاح‌طلبان حمایت کردند و خود او نیز در آخرین نمازجمعه‌ای که در تهران اقامه کرد و پس از آن در محافل خصوصی از سرکوب معترضان به نتایج این انتخابات انتقاد نمود. این انتقادات باعث شد که از سوی مراجع حکومتی به همسویی با معترضان متهم شده و به تدریج جایگاه امامت نمازجمعه، ریاست مجلس خبرگان و اثرگذاری بر دانشگاه آزاد اسلامی را از دست داده و فرزندانش نیز پست‌های خود را از دست بدهند.[۱۷]
رفسنجانی در سال ۱۳۹۲ چهارمین بار در انتخابات ریاست جمهوری نامزد شد که صلاحیت وی از سویشورای نگهبان احراز نشد. با پیروزی حسن روحانی در انتخابات، با توجه به رابطه نزدیک او با هاشمی رفسنجانی و حمایت هاشمی از وی در جریان انتخابات، خانواده هاشمی رفسنجانی به تدریج وجهه سیاسی خود را بازیافتند و به سرعت رئیس دانشگاه آزاد اسلامی تغییر کرد. او در سال‌های پایانی عمر خود توانست محبوبیتی مکدر در میان ایرانیان پیدا کند.[۱۸
اکبر هاشمی بهرمانی (زادهٔ ۳ شهریور ۱۳۱۳ در رفسنجان – درگذشتهٔ ۱۹ دی ۱۳۹۵ در تهران) مشهور بهعلی اکبر هاشمی رفسنجانی روحانی، مفسر قرآن،[۱۱][۱۲] و سیاستمدار ایرانی بود که به عنوان یکی از پرنفوذترین شخصیت‌های سیاسی جمهوری اسلامی شناخته می‌شد. او از ۲۵ مرداد ۱۳۶۸ تا ۱۲ مرداد ۱۳۷۶ چهارمین رئیس‌جمهور ایران بود. پیش از آن وی اولین رئیس مجلس شورای اسلامی از سال ۱۳۵۹ تا ۱۳۶۸ بود. در طول جنگ ایران و عراق رفسنجانی به عنوان نماینده رهبر در شورای عالی دفاع عملاً فرماندهی نیروهای مسلح را برعهده داشت. رفسنجانی از سال ۱۳۶۸ تا زمان مرگ، ریاست مجمع تشخیص مصلحت نظام را برعهده داشت. او در همه ادوار مجلس خبرگان رهبری نمایندگی استان تهران در آن را بر عهده داشت و از ۱۳۸۶ تا ۱۳۹۰، دومین رئیس این مجلس بود. رفسنجانی رئیس هیئت مؤسس و هیئت امنای دانشگاه آزاد اسلامی بود.
او یکی از تعیین‌کننده‌ترین سیاستمداران ایران، دوست نزدیک سید علی خامنه‌ای و شاگرد روح‌الله خمینی و رازدار اسرار جمهوری اسلامی ایران بود.[۱۳] در دوران ریاست جمهوری، با توجه به شرایط جدید کشور و پایان جنگ، وی اقدامات گسترده‌ای را در طی ۸ سال در ابعاد گوناگون سیاست داخلی و خارجی، اقتصادی، نظامی و امنیتی انجام داد که به دوران سازندگی شهرت یافته‌است. مهم‌ترین اولویت دولت در این دوران، بازسازی زیربناهای اقتصادی و صنعتی کشور، توسعه روابط سیاست خارجی و اصلاح ساختار مدیریتی کشور بود که انتقاداتی را نیز در پی داشته‌است.[۱۴] در این دوران، تعداد بسیار زیادی سد، کارخانه و کارگاه تولید صنعتی راه‌اندازی، و در بسیاری از شهرهای کوچک و بزرگ دانشگاه ساخته گردید.[۱۵] این اصلاحات، تبعاتی منفی نظیر افزایش تورم و قیمت مسکن در دورهٔ اول ریاست جمهوری وی را نیز در پی داشت. ثبت رکورد تورم ۴۹٫۴ درصدی بالاترین رقم تورم ثبت شده در تاریخ ایران، در دوران ریاست جمهوری وی اتفاق افتاد.[۱۶]
هاشمی رفسنجانی در سال ۱۳۸۴ در انتخابات ریاست جمهوری شرکت کرد که در دور دوم از محمود احمدی‌نژاد شکست خورد. در جریان انتخابات ریاست جمهوری ۱۳۸۸ هاشمی و خانواده او به وضوح ازمیرحسین موسوی، کاندیدای اصلاح‌طلبان حمایت کردند و خود او نیز در آخرین نمازجمعه‌ای که در تهران اقامه کرد و پس از آن در محافل خصوصی از سرکوب معترضان به نتایج این انتخابات انتقاد نمود. این انتقادات باعث شد که از سوی مراجع حکومتی به همسویی با معترضان متهم شده و به تدریج جایگاه امامت نمازجمعه، ریاست مجلس خبرگان و اثرگذاری بر دانشگاه آزاد اسلامی را از دست داده و فرزندانش نیز پست‌های خود را از دست بدهند.[۱۷]
رفسنجانی در سال ۱۳۹۲ چهارمین بار در انتخابات ریاست جمهوری نامزد شد که صلاحیت وی از سویشورای نگهبان احراز نشد. با پیروزی حسن روحانی در انتخابات، با توجه به رابطه نزدیک او با هاشمی رفسنجانی و حمایت هاشمی از وی در جریان انتخابات، خانواده هاشمی رفسنجانی به تدریج وجهه سیاسی خود را بازیافتند و به سرعت رئیس دانشگاه آزاد اسلامی تغییر کرد. او در سال‌های پایانی عمر خود توانست محبوبیتی مکدر در میان ایرانیان پیدا کند.[۱۸