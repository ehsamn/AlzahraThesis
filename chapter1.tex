\chapter{عنوان فصل اول }
%\thispagestyle{empty}
% بهتر است ابتدای هر فصل مقدمه نوشته شود 
\section{مقدمه}
این فصل
\inpdic{فصل}{Chapter}
 شامل برخی از پیشنیازهای  مورد نیاز برای تهیه‌ی یک  \inpdic{رساله}{Dissertation} است.
\section{عنوان بخش دوم}
واژهٔ دیگری به واژه‌نامه اضافه می‌کنیم: \inind{پوش محدب}{convex hull}
برای امتحان اضافه کردن یک منبع فارسی به این استناد می‌دهم \cite{irscholar494611}. در منبع   \cite{glasner2007enveloping} و \cite{diestel2012sequences} برای این که ببینم پرانتز درست کار می‌کند یا نه (متنی در پرانتز) حالا یک متن انگلیسی در پرانتز (\lr{Test})
یک شرکت   تولیدی می‌خواهد برای برآورده کردن نیاز خرده‌فروش‌ها انبارهایی در سطح شهر بسازد. از قبل محل‌هایی که امکان برپایی انبار در آن‌ها وجود دارد، پیش‌بینی شده‌اند. تاسیس هر انبار هزینه‌‌ای ثابت را برای شرکت درپی دارد. فرض کنید که شرکت می‌تواند با پرداخت هزینه‌ی ثابت اولیه، برای برخی از مشتریان خاص، کالاها را به طور مستقیم (بدون استفاده از انبار) ارسال کند. بنابراین، این دسته از مشتریان می‌توانند تقاضای خود را  بدون  پرداخت هزینه‌ی جابجایی از انبار، به طور مستقیم از شرکت دریافت کنند. هدف نهایی این است که توزیع کالا به گونه‌ای انجام شود که مجموع هزینه‌ی ثابت برپایی، هزینه‌ی ثابت انتقال کالا به مشتریان خاص و هزینه‌ی ارسال کالا از انبارها به خرده‌فروش‌ها کمینه شود.  این مسأله را مکان‌یابی تسهیلات با نقاط تقاضای \inpdic{خودخدمت‌گزار}{self-serving}   (\lr{UFLP-SS}) می‌نامیم. از این دیدگاه نیز می‌توان مسأله را طرح کرد که برخی از نقاط تقاضا این امکان را دارند که مرکز خدمت‌گزار هم باشند. به چنین مراکز خدمت‌گزاری، مرکز \inpdic{طرف تقاضا}{client side}  و به نقاط تقاضایی که این امکان برای آن‌ها وجود دارد  نقطه‌ی تقاضای خودخدمت‌گزار  می‌گوییم. 

مدل‌سازی ریاضی این مسأله  مشابه مسأله‌ی مکان‌یابی  تسهیلات است. فرض کنید $I=\{1,\cdots,m\}$ مجموعه‌ی نقاط تقاضا، $J=\{1,\cdots,n\}$ مجموعه‌ی مراکز خدمت‌گزار و $K\sseq I$ مجموعه‌ی نقاط تقاضای خودخدمت‌گزار و  $L$ سایر نقاط تقاضا هستند. نقاط تقاضای خودخدمت‌گزار باید همه‌ی نیاز خود را یا از طریق مرکز تاسیس شده در مکان خود یا سایر مراکز دریافت کنند. از این رو، برای هر نقطه‌ی تقاضای خودخدمت‌گزار $k\in K$،  $z_{k}$ را متغیر دودویی تعریف می‌کنیم که یک بودن آن نشان دهنده‌ی این است که نقطه‌ی تقاضای $k$ از مرکز  طرف تقاضای خود سرویس‌دهی می‌شود؛ برای هر $i\in I$ و $j\in J$، متغیر $x_{ij}$ تنها هنگامی برابر یک است که تقاضای نقطه‌ی $i$ توسط مرکز $j$ برآورده شود؛  برای  $j\in J$، متغیر دودویی $y_j$ برابر یک است هرگاه در مکان $j$ یک مرکز خدمت‌رسان تاسیس شود. در نتیجه، مدل بهینه‌سازی خطی با اعداد صحیح برای مسأله به صورت زیر نوشته می‌شود:
 \begin{align}
 \min  \quad    & \sum_{i\in I}\sum_{j\in J} c_{ij}x_{ij} +  \sum_{j\in J} f_j y_j + \sum_{k\in K} g_kz_k   \label{eqn:uflp-ss-obj}\\
 \st \quad      & \sum_{j\in J} x_{kj} + z_k = 1 ,  \; k\in K, \label{eqn:uflp-ss-c1}\\
                & \sum_{j\in J}  x_{lj} = 1 ,  \; l\in L, \label{eqn:uflp-ss-c2}\\
                & x_{ij} - y_j \leq 0,  \; i\in I, \; j\in J, \label{eqn:uflp-ss-c3}\\
                & z_k,y_j, x_{ij} \in \BB,\; i\in I, j\in J, k\in K,  \label{eqn:uflp-ss-bin-const}
 \end{align}
که در آن،  $g_k$ هزینه‌ی ثابت تاسیس مرکز طرف تقاضا، $f_j$ هزینه‌ی ثابت تاسیس یک مرکز و $c_{ij}$ هزینه‌ی سرویس‌دهی نقطه‌ی تقاضای $i$ توسط مرکز $j$ است. قید‌های \eqref{eqn:uflp-ss-c1} و  \eqref{eqn:uflp-ss-c2}   برای اطمینان از تامین نیازهای نقاط تقاضا هستند. قید‌های \eqref{eqn:uflp-ss-c3}   باعث می‌شوند که اگر مرکزی در نقطه‌ی $j$ تاسیس نشود، آن‌گاه هیچ نقطه‌ی تقاضایی از آن خدمات دریافت نکند. 

اگر $K=\emptyset$، آن‌گاه مسأله به یک \lr{UFLP} تبدیل می‌شود. بنابراین \lr{UFLP-SS} تعمیمی ساده از مسأله‌ی مکان‌یابی  تسهیلات و در نتیجه  \lr{NP}-سخت
 است.

واسکو\LTRfootnote{Vasko} و همکاران \cite{VNSW2003} مسأله‌ی مکان‌یابی تسهیلات با \inpdic{پوشش جزیی}{partial coverage}  را معرفی کردند که دارای یک مدل بهینه‌سازی با اعداد صحیح مشابه با \lr{UFLP-SS} است. آن‌ها با تبدیل مسأله به یک \lr{UFLP} آن را با روش فراز دوگان ارلنکاتر حل کردند. مسأله‌ی آن‌ها همان مکان‌یابی  تسهیلات است با این فرض اضافی که می‌توان با پرداخت جریمه، نیاز یک نقطه‌ی تقاضا را برآورده نکرد. در واقع، مراکز تقاضایی که برای آن‌ها جریمه پرداخت می‌شوند، معادل مراکز خودخدمت‌گزار در \lr{UFLP-SS} هستند. 


چاریکار\LTRfootnote{Charikar} و همکاران \cite{CKMN2001} مسأله‌ای با عنوان  مکان‌یابی با مراکز دورافتاده   را بررسی و الگوریتم‌های تقریبی برای آن ارایه کردند که مدل بهینه‌سازی با اعداد صحیح آن مشابه با \lr{UFLP-SS} است. آن‌ها بیان می‌کنند که در برخی از کاربردهای مسایل مکان‌یابی، وجود   مراکزی که فاصله‌ی معنی‌داری با سایرین دارند، موجب کاهش کیفیت جواب می‌شود. در نتیجه،  اجازه می‌دهند که با پرداخت جریمه‌ای نیاز این نقاط دورافتاده برآورده نشود. این مسأله با عنوان مکان‌یابی با جریمه\LTRfootnote{penalty} نیز شناخته می‌شود \cite{XX2005}.

 بایو\LTRfootnote{Ba\"{i}ou} و باراهونا\LTRfootnote{Barahona}   \cite{BB2009} فضای شدنی این مسأله را در حالتی که قیود تساوی  با نامساوی جایگزین شوند، بر روی یک گراف جهت‌دار بررسی کردند. آن‌ها بیان کردند که تحت چه شرایطی این مجموعه دارای نقاط رأسی با اعداد صحیح است.
 
 برخی از حالت‌های خاص مسأله‌ی مکان‌یابی تسهیلات را  نیز  می‌توان به \lr{UFLP-SS} تبدیل کرد. به عنوان مثال، در مسأله‌ی مکان‌یابی تسهیلات فرض کنید که مکان‌های بعضی از مراکز خدمت‌گزار از پیش معلوم هستند و قرار است که چند مرکز جدید به مراکز پیشین اضافه شوند. برای بیان دقیق‌تر، فرض کنید که $I$ مجموعه‌ی نقاط تقاضا، $J$ مجموعه‌ی مکان‌های ممکن برای تاسیس مراکز جدید و $J^P$  مجموعه‌ی مراکزی باشد که مکان‌هایشان از پیش معلوم شده‌اند. قید مربوط به تامین تقاضاها در این حالت را می‌توان به صورت زیر نوشت:
 \begin{align*}
 \sum_{j\in J} x_{ij} +  \sum_{j\in J^P} x_{ij}= \sum_{j\in J} x_{ij} + z_i= 1,
 \end{align*}
که در آن، $z_i$ یک متغیر دودویی برابر با مجموع $\sum_{j\in J^P} x_{ij} $ است.   $z_i$ وقتی برابر یک است که نقطه‌ی تقاضای  $i$ توسط مراکز از پیش ساخته شده تامین شود. در این حالت، مرکزی انتخاب می‌شود که کم‌ترین هزینه  $\min_{j\in J^P} c_{ij}$ را داشته باشد. در نتیجه، می‌توان هزینه‌ی  جابه‌جایی خدمات به نقطه‌ی  تقاضا   را بدین صورت زیر اصلاح کرد: $\sum_{j\in J} c_{ij} x_{ij} + z_i g_i$  که در آن، $g_i = \min_{j\in J^P} c_{ij}$. این نشان می‌دهد که مسأله‌ی اخیر با \lr{UFLP-SS} معادل است. 

\lr{UFLP-SS} در برخی از روش‌ها به عنوان یک زیرمسأله ظاهر می‌شود. در  مدل بهینه‌سازی با اعداد صحیح  \lr{UFLP-SS} فرض کنید $K=I$.  اگر متغیر $z_i$ را به عنوان یک متغیر کمبود  در نظر بگیریم، آن‌گاه مدل بهینه‌سازی با اعداد صحیح به صورت زیر در می‌آید:
 \begin{align}
\min  \quad    & \sum_{i\in I}\sum_{j\in J} c_{ij}x_{ij} +  \sum_{j\in J} f_j y_j   + \sum_{k\in K} g_k\left(1-\sum_{j\in J}x_{kj} \right) \\
 \st \quad      & \sum_{j\in J} x_{ij} \leq 1 ,  \; i\in I, \\
                & x_{ij} - y_j \leq 0,  \; i\in I, \; j\in J, \\
                & y_j, x_{ij} \in \BB, \; i\in I, \; j\in J.
 \end{align}
این مسأله‌ای است که بلترن-رویو\LTRfootnote{Beltran-Royo} و همکاران در \cite{BVA2012} از آن در روش خود به عنوان زیرمسأله برای حل \lr{UFLP} استفاده می‌کنند.
\section{روش حل}
در این بخش یک روش فراز دوگان و یک روش تنظیم دوگان مانند آنچه توسط ارلنکاتر در \cite{E1978}  آمده است، ارایه می‌کنیم. هدف اصلی در این روش، حل مسأله‌ی رهاسازی برنامه‌ریزی خطی است. 
\subsection{روش حل اول}

\begin{figure}[t]

\caption{متنی برای عنوان شکل}
\end{figure}

با رهاسازی  صفر و یک بودن متغیرهای $x_{ij}$، $z_k$ و $y_j$، رهاسازی برنامه‌ریزی خطی مسأله به صورت زیر به دست می‌آید:
 \begin{align}
\min  \quad    & z_p(\bx,\by,\bz) = \sum_{i\in I}\sum_{j\in J} c_{ij}x_{ij} +  \sum_{j\in J} f_j y_j  + \sum_{k\in K} g_kz_k \label{eqn:uflpss-lp-obj}\\
 \st \quad      & \sum_{j\in J} x_{kj} + z_k = 1,  \quad  k\in K,  \label{eqn:uflpss-lp-c1}\\
                & \sum_{j\in J} x_{lj} = 1,  \quad l\in L, \label{eqn:uflpss-lp-c2}\\
                &  -x_{ij} + y_j \geq 0,  \quad i\in I, j\in J,  \label{eqn:uflpss-lp-c3}\\
                & z_k,y_j, x_{ij} \geq 0, \quad i\in I, j\in J, k\in K. \label{eqn:uflpss-lp-c4}
 \end{align}
دوگان این مسأله به صورت زیر نوشته می‌شود:
 \begin{align}
 \max  \quad   &  \sum_{k\in K} u_k + \sum_{l\in L} v_l  \label{eqn:uflpss-weak-dual-obj}\\
 \st \quad     &   u_k - w_{kj} \leq c_{kj},        \quad  k\in K, j\in J,  \label{eqn:uflpss-weak-dual-c1}\\
               &   v_l - w_{lj} \leq c_{lj},          \quad l\in L,  j\in J,       \label{eqn:uflpss-weak-dual-c2}\\
               &   u_k \leq g_k,                      \quad  k\in K,       \label{eqn:uflpss-weak-dual-c3}\\
               &   \sum_{i\in I} w_{ij} \leq f_j,  \quad j\in J,       \label{eqn:uflpss-weak-dual-c4}\\
               & w_{ij} \geq 0, \quad i\in I, j\in J.                      \label{eqn:uflpss-weak-dual-c5}
 \end{align}
با توجه به قیود \eqref{eqn:uflpss-weak-dual-c1}، \eqref{eqn:uflpss-weak-dual-c2} و \eqref{eqn:uflpss-weak-dual-c5} داریم:
\begin{align}
& w_{kj} \geq (u_k-c_{kj})^+, \quad k\in K, j\in J, \label{eqn:uflpss-wdual-w1}\\
& w_{lj} \geq (v_l-c_{lj})^+, \quad l\in L, j\in J, \label{eqn:uflpss-wdual-w2}
\end{align}
که در آن، $(a)^+=\max\{0,a\}$. از آن‌جا که متغیرهای $w_{ij}$ در تابع هدف ظاهر نمی‌شوند، پس می‌توان فرض کرد  که در جواب بهینه، رابطه‌های \eqref{eqn:uflpss-wdual-w1} و \eqref{eqn:uflpss-wdual-w2} به صورت تساوی برقرارند، یعنی $w_{kj} = (u_k-c_{kj})^+$، برای هر $k\in K$، و $j\in J$، و $w_{lj} = (v_l-c_{lj})^+$، برای هر $l\in L$ و $j\in J$. با جایگذاری متغیرهای $w_{ij}$ در قید‌ها، مسأله‌ی زیر بدست می‌آید که مشابه با فرم خلاصه‌شده‌ی دوگان برای \lr{UFLP} است
 \begin{align}
 \max  \quad   &  z_d(\bu,\bv) = \sum_{k\in K} u_k + \sum_{l\in L} v_l  \label{eqn:uflpss-con-wdual-obj} \\
 \st \quad     &   \rho_j(\bu,\bv)\leq 0,        \quad j\in J,  \label{eqn:uflpss-con-wdual-c1}\\
               &   u_k \leq g_k,                      \quad k\in K, \label{eqn:uflpss-con-wdual-c2}
 \end{align}
 که در آن،
 \begin{align*}
 \rho_j(\bu,\bv) = \sum_{k\in K} (u_k-c_{kj})^+ +  \sum_{l\in L} (v_l-c_{lj})^+ - f_j.
 \end{align*}
 جفت جواب شدنی اولیه-دوگان به صورت $(\bx,\by,\bz)$  و $(\bu,\bv)$ بهینه است، هرگاه در شرایط کمبود تکمیلی صدق کند، یعنی داشته باشیم:
\begin{align}
 x_{kj} ( u_k - c_{kj} )^- = 0, \; &\; k\in K, j\in J,                     \label{eqn:uflp-ss-cs1}\\
 x_{lj} ( v_l - c_{lj} )^- = 0, \; & \;l\in L, j\in J,                    \label{eqn:uflp-ss-cs2}\\
 z_k (u_k - g_k) = 0, \; & \;k\in K,                            \label{eqn:uflp-ss-cs3}\\
 y_j \rho_j(\bu,\bv)  = 0, \; & \;j\in J,                                    \label{eqn:uflp-ss-cs4}\\
 ( u_k - c_{kj} )^+( y_j - x_{kj} ) =0, \; & \;k\in K, j\in J,  \label{eqn:uflp-ss-cs5}\\
 ( v_l - c_{lj} )^+( y_j - x_{lj} ) =0, \; & \;l\in L, j\in J,  \label{eqn:uflp-ss-cs6}
\end{align}
که در آن، $(a)^-=\min\{0,a\}$. رابطه‌های \eqref{eqn:uflp-ss-cs1} و \eqref{eqn:uflp-ss-cs2} با توجه به   $a-(a)^+=(a)^-$ بدست آمده‌اند. اینک، با در دست داشتن یک جواب شدنی دوگان مانند $(\bu,\bv)$، می‌توان یک جواب اولیه شدنی مانند $(\bx,\by,\bz)$  با درایه‌های  عدد صحیح تولید کرد که در شرایط کمبود تکمیلی \eqref{eqn:uflp-ss-cs1} تا \eqref{eqn:uflp-ss-cs4} صدق کند.  بدین منظور، دو مجموعه‌ی $\bar J(\bu,\bv)$ و $\bar K(\bu)$ را به صورت زیر تعریف کنید: 

\begin{figure}[t]

\caption{متنی برای عنوان شکل}
\end{figure}
